%!TEX root = ../template.tex
%%%%%%%%%%%%%%%%%%%%%%%%%%%%%%%%%%%%%%%%%%%%%%%%%%%%%%%%%%%%%%%%%%%%
%% abstract-it.tex
%% NOVA thesis document file
%%
%% Abstract in Italian
%%%%%%%%%%%%%%%%%%%%%%%%%%%%%%%%%%%%%%%%%%%%%%%%%%%%%%%%%%%%%%%%%%%%

\typeout{NT FILE abstract-it.tex}%


\textbf{Questa è una traduzione "Google Translate" della versione inglese! Riparazioni e correzioni sono benvenute!}

Indipendentemente dalla lingua in cui è scritta la tesi, di solito ci sono almeno due abstract: un abstract nella stessa lingua del testo principale e un altro abstract in un'altra lingua.

L'ordine degli abstract varia a seconda della scuola. Se la tua scuola ha norme specifiche riguardanti l'ordine degli abstract, il template \gls{novathesis} (\LaTeX) le rispetterà. Altrimenti, la regola predefinita nel template \gls{novathesis} è di avere in primo luogo l'abstract nella \emph{la stessa lingua del testo principale}, e poi l'abstract nell'\emph{l'altra lingua}. Ad esempio, se la tesi è scritta in portoghese, l'ordine degli abstract sarà prima in portoghese e poi in inglese, seguito dal testo principale in portoghese. Se la tesi è scritta in inglese, l'ordine degli abstract sarà prima in inglese e poi in portoghese, seguito dal testo principale in inglese.
%
Tuttavia, questo ordine può essere personalizzato aggiungendo uno dei seguenti al file \verb+5_packages.tex+.

\begin{verbatim}
    \ntsetup{abstractorder={<LANG_1>,...,<LANG_N>}}
    \ntsetup{abstractorder={<MAIN_LANG>={<LANG_1>,...,<LANG_N>}}}
\end{verbatim}

Ad esempio, per un documento principale scritto in tedesco con abstract scritti in tedesco, inglese e italiano (in questo ordine) utilizzare:
\begin{verbatim}
    \ntsetup{abstractorder={de={de,en,it}}}
\end{verbatim}

Per quanto riguarda i contenuti, gli abstract non dovranno superare una pagina e potranno rispondere alle seguenti domande (è fondamentale adeguarsi alle prassi abituali della propria area scientifica):

\begin{itemize}
  \item Qual è il problema?
  \item Perché questo problema è interessante/stimolante?
  \item Qual è l'approccio/soluzione/contributo proposto?
  \item Quali risultati (implicazioni/conseguenze) derivano dalla soluzione?
\end{itemize}

% Palavras-chave do resumo em Italiano
\keywords{
  Una parola chiave \and
  Un'altra parola chiave \and
  Ancora un'altra parola chiave \and
  Una parola chiave in più \and
  L'ultima parola chiave
}
