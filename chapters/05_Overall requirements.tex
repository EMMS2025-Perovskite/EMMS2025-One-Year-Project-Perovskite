\chapter{Overall System Requirements}
\label{chpt:overall-system-requirements}

This section has been prepared in accordance with the methodology defined in ECSS-E-ST-10-06C – Technical Requirements Specification. It outlines the general system requirements, which serve as a foundation for future design efforts and the development of more detailed, subsystem-specific requirements. At this stage, the focus shifts from strict verifiability to providing high-level guidance. Preliminary subsystem requirements are included in Appendix~B.

\section{Requirements for the Scientific Community}
The primary beneficiary group, as identified in the Use Cases, is the scientific community. 
Accordingly, the following requirements focus on ensuring high-fidelity time-series data of perovskite solar cell performance under LEO conditions, 
enabling accurate quantification of their degradation and recovery.
Tabela \ref{tbl:requirements-science} summarizes the requirements directed at the scientific stakeholders.

\begin{table}[!ht]
    \centering
    \small
    \caption{First part of the overall system requirements related to the science community}
    \label{tbl:requirements-science}
    \begin{tblr}{%
        hlines,
        vlines,
        row{1} = {font=\bfseries},
        column{1} = {halign=c},
    }
        ID   & Type         & Description                                                                                                            \\
        OR1  & Functional   & The payload subsystem shall perform a set of measurements per orbit on each of the perovskite solar cell samples under solar illumination: \\
             &              & a) the open-circuit voltage (Voc),                                                                                        \\
             &              & b) short-circuit current (Isc),                                                                                           \\
             &              & c) maximum power (Pmax).                                                                                                   \\
        OR2  & Functional   & All the measurements must be time-stamped and logged onboard.                                                             \\
        OR3  & Functional   & The data acquisition hardware shall capture voltage and current measurements with a maximum error of 10\%.                  \\
        OR4  & Functional   & The system shall be able to apply a controlled electrical bias of the set voltage value up to 5\,V to the perovskite solar cells. \\
        OR5  & Functional   & The AOCS shall include sensors and actuators that ensure sun-pointing capabilities.                                        \\
        OR6  & Functional   & The AOCS shall orient the satellite such that the perovskite panels receive sunlight with up to 10 degrees of precision.  \\
        OR7  & Functional   & The COMMS shall downlink the logged experiment data in a format compliant with GS and support uplink of commands.          \\
        OR8  & Verification & The perovskite solar cells shall be characterised pre-flight.                                                             \\
        OR9  & Design       & The number of perovskite solar cells should allow for a safe and representative experiment.                                \\
        OR10 & Logistics Support & The research results shall be published in a scientific journal or academic conference.                                 \\
    \end{tblr}
\end{table}

\section{Requirements for the Industry Community}

Furthermore, the space industry and perovskite manufacturers will utilise the results to inform and optimise future material and device designs.
They require quantified degradation rates, clear evidence of any self-healing behaviour, and detailed technical documentation.
Beyond the requirements mentioned beforehand, address the needs of industrial stakeholders.

\begin{table}[!ht]
    \centering
    \small
    \caption{Second part of the overall system requirements related to the industry community}
    \label{tbl:requirements-industry}
    \begin{tblr}{%
        hlines,
        vlines,
        row{1} = {font=\bfseries},
        column{1} = {halign=c},
    }
        ID    & Type               & Description                                                                                                      \\
        OR9   & Logistics Support  & The subcontractor for the development of the perovskite solar cells shall be an institution oriented toward research and development. \\
        OR10  & Mission            & The mission shall demonstrate the usage of perovskite solar cells in space conditions.                                                                                        \\
        OR11  & Environmental      & The environmental parameters (e.g.\ ionising radiation, temperature, solar irradiance) shall be recorded concurrently.                                                       \\
        OR12  & Product Assurance  & The development and testing procedures of the payload and EPS shall be documented, including design specifications, qualification tests, and integration processes.           \\
        OR13  & Logistics Support  & The data format, metadata structure, and analysis pipeline shall conform to open standards, enabling direct use.                                                             \\
    \end{tblr}
\end{table}

\section{Requirements for the Student Community}

The mission also aims to engage and benefit the student and young professional community. 
As an integral part of the space sector’s future workforce, they gain valuable educational and practical insights from the mission’s outcomes. 
The project will provide multi-level resources tailored to varying degrees of expertise, from introductory overviews to advanced technical descriptions. 
This approach ensures the effective transfer of knowledge and facilitates the inclusion of perovskite research in academic curricula, student projects, 
and university-led initiatives. Requirements described in Table \ref{tbl:requirements-student} support these objectives.

\begin{table}[!ht]
    \centering
    \small
    \caption{Third part of the overall system requirements related to the student community}
    \label{tbl:requirements-student}
    \begin{tblr}{%
        hlines,
        vlines,
        row{1} = {font=\bfseries},
        column{1} = {halign=c},
    }
        ID    & Type              & Description                                                                                                                                    \\
        OR14  & Logistics Support & The mission shall include roles for student team members in all major system engineering, project management and research activities.                 \\
        OR15  & Logistics Support & Student involvement shall produce measurable outcomes, such as theses, reports, or publications.                                                   \\
        OR16  & Design            & The system design shall enable assembly, integration, and testing by the student team under supervision.                                            \\
        OR17  & Logistics Support & The mission shall provide open or university-accessible research data acquired during its operation in a public repository or portal.                \\
        OR18  & Logistics Support & The mission shall provide an open or university-accessible design and testing documentation archived in a public repository or portal.               \\
    \end{tblr}
\end{table}

\section{Requirements Supporting ESA Objectives}

The requirements outlined below demonstrate how the mission aligns with the strategic objectives of the European Space Agency (ESA). 
Through this endeavour, we aim to contribute meaningfully to ESA’s vision for the space sector. By adopting open data policies and employing a cost-effective CubeSat design, 
the project demonstrates how small satellite platforms can efficiently validate cutting-edge technologies while maintaining compliance with debris mitigation guidelines 
and ECSS standards. Requirements in the following table pertain to ESA-related goals.

\begin{table}[!ht]
    \centering
    \small
    \caption{Fourth part of the overall system requirements related to ESA objectives}
    \label{tbl:requirements-esa}
    \begin{tblr}{%
        hlines,
        vlines,
        row{1} = {font=\bfseries},
        column{1} = {halign=c},
    }
        ID    & Type           & Description                                                                                                                       \\
        OR19  & Mission        & The mission shall validate perovskite reverse-bias treatment and assess its influence, helping increase its TRL.                     \\
        OR20  & Logistics Support & The mission shall enable technology transfer and industrial uptake by European SMEs.                                                 \\
        OR21  & Design         & The system shall serve as a model for an economically viable platform to test PSC.                                                   \\
        OR22  & Mission        & The mission shall ensure end-of-life compliance by deorbiting the CubeSat within 25 years, in accordance with debris mitigation.   \\
        OR23  & Product Assurance & The system shall comply with ECSS-E-ST-10-06C and all applicable standards throughout design, development, and operations.           \\
    \end{tblr}
\end{table}

