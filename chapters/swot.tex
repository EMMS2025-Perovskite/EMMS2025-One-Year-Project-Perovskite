\chapter{Swot}
% You can label chapter to refer to it later in the text, e.g. in the
% later chapter.
\label{chpt:sample-chapter}

TEST
The entire thesis should consist of \emph{chapters}, which are divided
into \emph{sections}. Sections can further
contain \emph{subsections}. This entire structure is visible
in the table of contents. Additionally, LaTeX allows dividing subsection
points into parts called \emph{subsubsections}, which are, with the settings
of this template, invisible in the table of contents. However, they can still
be useful for breaking longer subsection points into logical parts.

\section{Sample Section}

\subsection{Sample Subsection}

\subsubsection{Figures}

\begin{figure}[!htb]
    \centering
    \includegraphics[width=0.25\linewidth]{drawing_example}
    \caption{Logo of the LaTeX book on Wikibooks
    \cite{book:latex}}
    \label{fig:sample-figure}
\end{figure}

Figures and tables should be included according to the template, always specifying:

\begin{itemize}
    \item centered position of the figure: \mono{{\textbackslash}centering};
    \item dimensions of the figure and the name of the source file;
    \item caption under the figure with citation, if it comes
    from an external source;
    \item a label, preferably starting with \emph{fig}, since
    the namespace defined in \mono{{\textbackslash}label} is
    global.
\end{itemize}

As a parameter in \mono{{\textbackslash}includegraphics}, provide the file name without extension. The image file should be placed
in the \emph{figures} folder or another one configured in the preamble.
It’s a good idea to use vector formats to preserve high
print quality. Unfortunately, LaTeX has very limited
SVG support, so a common solution is to convert vector graphics
to PDF format.

Every figure should be referenced in the text. Here, the figure serves as an example—
the logo of the LaTeX book \ref{fig:sample-figure}, which
is available on Wikibooks.

\subsubsection{Tables}

\begin{table}[!ht]
    \centering
    \small
    \caption{Table Title}
    \label{tbl:table-label}
    \begin{tblr}{%
        hlines,%
        vlines,%
        row{1}={font=\bfseries},%
        column{1}={halign=c},%
    }%
        No. & Description                        \\
        1   & Basic table with two columns       \\
    \end{tblr}
\end{table}


The \emph{tabularray} package allows easy creation of both simple
and complex tables. Every table should also be referenced
in the text. Here, a sample table is defined as
\ref{tbl:table-label}. Note that tables and figures
are numbered independently, so they may share the same number. Therefore,
context should always make clear whether it refers to a table or
a figure.

\subsubsection{Citations and Bibliography}

Bibliographic entries should be added to the
\emph{config/bibliography.bib} file using the biblatex standard. They will appear
in the bibliography only after being cited in the text
using \mono{{\textbackslash}cite}. As an example, we can cite
the aforementioned LaTeX book \cite{book:latex}.

\subsubsection{Equations}

Equations are easiest to define using LaTeX’s built-in
\mono{{\textbackslash}equation} environment, which supports
inserting symbols in math mode. Equations can also be labeled
to refer to them in the text, though it's not required. In this case,
we can refer to equation \ref{eq:sample-equation}.

\begin{equation} \label{eq:sample-equation}
    \Delta p = K_s (\frac{1}{T_0} - \frac{1}{T_s}) h
\end{equation}

where:

\begin{conditions}
    \Delta p &  pressure difference $[Pa]$ \\
    K_s      &  conversion coefficient equal to $3460$ \\
    T_0      &  outdoor air temperature $[K]$ \\
    T_s      &  indoor air temperature $[K]$ \\
    h        &  distance from the neutral plane $[m]$ \\
\end{conditions}

Each equation should use symbols that are clearly
defined. A good tool for this is the \mono{{\textbackslash}conditions}
environment, defined in this template.

\subsubsection{Best Practices}

By default, LaTeX uses a larger space between two sentences than between
words within a sentence, which is a common
Anglo-Saxon typographic practice also seen in Poland.
An alternative is the French approach, which uses equal
spacing between words and between sentences. In LaTeX, the latter
behavior can be enabled using the
\mono{{\textbackslash}frenchspacing} option.

If you don’t enable it, remember that all abbreviations
in the middle of a sentence ending with a period should be handled specially
so as not to trigger this behavior. You can do this
by adding a backslash after the period where the space
should not be increased. For example, the abbreviation for *for example*:
\mono{e.g.\textbackslash}

An alternative method is to add a non-breaking space, done
by replacing the space with a tilde \textasciitilde, but it should
only be used where necessary, typically to avoid widows
at line ends.

\subsubsection{Summary}

This chapter explains the basic rules of working with the template using real
examples, so it’s worth reviewing both the source code and
the generated PDF. Additionally, reviewing the configuration files
is recommended, where individual settings are accompanied by comments.
